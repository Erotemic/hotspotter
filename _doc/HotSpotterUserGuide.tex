\documentclass[a4paper,10pt]{article}
\usepackage[utf8]{inputenc}
\usepackage{graphicx}
\usepackage{enumerate}
\usepackage{listings}
\usepackage{multirow}
\usepackage{pdfpages}
\usepackage{amsmath}
\usepackage{amsthm}
\usepackage[margin=1.3in]{geometry}
\usepackage[backref=true,pagebackref=true,breaklinks=true,letterpaper=true,colorlinks,bookmarks=true,bookmarksnumbered=true,bookmarksopen=true,hyperfigures=true]{hyperref}
\usepackage{times}
\usepackage{textcomp}
\usepackage{fancyvrb}
% Uncomment to disallow hyphenated line breaks
\usepackage[none]{hyphenat}
\raggedright


\newcommand{\HRule}{\rule{\linewidth}{0.5mm}}
\newcommand{\eps}{\varepsilon}
\newcommand{\ssum}{\displaystyle\sum}

% MACRO DEFINITIONS
\newcommand{\developeremail}{{\tt hotspotter.ir@gmail.com}}


\begin{document}
\begin{titlepage}
    \topskip0pt
    \vspace*{\fill}

    \begin{center}

        \HRule \\[0.4cm]
        { \huge \bfseries HotSpotter User Guide}\\[0.4cm]
        \HRule \\[1.5cm]

        Jon Crall (crallj@rpi.edu)\\
        Jason Parham (parhaj@rpi.edu) \\
        Chuck Stewart (stewart@cs.rpi.edu) \\ 

        \medskip

        Department of Computer Science \\
        Rensselaer Polytechnic Institute

    \end{center}

    \vspace*{\fill}
\end{titlepage}

\DeclareRobustCommand{\cmdkey}{\raisebox{-.035em}{\includegraphics[height=.75em]{command}}}

\section{Usage}  This document presumes that you have downloaded and installed
the Windows or Mac version of the HotSpotter sofware, and describes the basic
steps of running the program to identify individual animals. The instructions
are primarily focused on the mac version of the software, but adaptation to
the windows version is easy. Just know that the control key ({\tt Ctrl}) on windows
is equivalent to the command key ({\tt Cmd} = \cmdkey) on the mac. This guide will use the
mac notation.

    \subsection{Opening the Program}
    When HotSpotter is first run, the program prompts the user to open a
    database or create a new one.  In each succeeding run, it will start by
    opening the previous database. 

    % Macros for displaying Mac/Windows Control Keys and arrows instead of ->
    \renewcommand{\textrightarrow}{$\rightarrow$}
    %\newcommand{\textcmd}{Cmd}
    \newcommand{\textcmd}{\cmdkey}
    %[fontsize=\footnotesize,frame=single,commandchars=\\\{\}]

    To Open or Create a new Database: \\
    \begin{Verbatim}[commandchars=\\\{\}]
    File \textrightarrow Open Database [(\textcmd+O)]
    \end{Verbatim}

    \begin{center}
        \includegraphics[scale=0.3]{img/start.png}
    \end{center}

    HotSpotter assumes that \textbf{empty folders} are given as new databases.
    HotSpotter can also read StripeSpotter databases by opening the
    StripeSpotter database's {\tt data} directory.
    
    \subsection{Importing images}
        In order to add one or more images to the database, run
        \begin{Verbatim}[commandchars=\\\{\}]
        File \textrightarrow Import Images [\textcmd+I] 
        \end{Verbatim}
        \begin{center}
            \includegraphics[scale=0.3]{img/addImage.png}
        \end{center}
        HotSpotter will will copy all selected images into its images directory.
        These are automatically added to the database and may be seen under ``Image View''

    \subsection{Defining Chips with ROIs and Orientation}
        Before identifying an animal in an image --- or, equivalently,
        finding other images that show the same animal ---- a region
        of interest (ROI) an orientation must be assigned.  (The
        sub-image extracted from an ROI is called a ``chip''.)  The
        ROIs must be specified first and this can be done either
        manually or automatically.  The automatic version is not very
        sophisticated: it simply assigns each image to be its own ROI.
        This is accomplished by
        \begin{Verbatim}[commandchars=\\\{\}]
        Convenience \textrightarrow Convert All Images to Chips
        \end{Verbatim}
        This option should be used in the relatively rare case that the
        animal occupies almost the entire image.

        The more common case is to specify the ROIs manually.  Multiple ROIs are allowed for each
        image.  Each ROI should include most of the body of the animal ---
        anything that might be a distinguishing feature --- so users should err on the
        side of making the ROI too large rather than too small.

        In order to specify an ROI, the Image Table should be
        highlighted and then the image should be selected.  Then,
        ``ROI mode'' must be entered:
        \begin{Verbatim}[commandchars=\\\{\}]
        Actions \textrightarrow Add ROI [A]
        \end{Verbatim}
        An ROI is selected by clicking two image points in the Plot
        Widget (on the right of the display) to specify opposite corners of the
        bounding box.

        \;

        In the case of an annotation mistake, an ROI can be reselected using 
        \begin{Verbatim}[commandchars=\\\{\}]
        Actions \textrightarrow Reselect ROI [R]
        \end{Verbatim}
        or removed entirely using 
        \begin{Verbatim}[commandchars=\\\{\}]
        Actions \textrightarrow Delete ROI [\textcmd+Delete]
        \end{Verbatim}

        The default orientation is horizontal, and this is set
        internally by HotSpotter.  This is usually sufficient when
        taking ``normal'' --- e.g.\ side-view --- pictures of standing
        animals, such as zebras or giraffes.  On the other hand, for overhead
        pictures of animals like frogs, specification of the
        orientation is \textbf{crucial for accurate recognition}.  The
        orientation is best determined by drawing an axis within the
        ROI of the animal in a way that can be repeated for each
        animal.  For frog images this is the spine.

        In order to specify an orientation other than the default
        (horizontal) orientation, the user must enter orientation
        mode:
        \begin{Verbatim}[commandchars=\\\{\}]
        Actions \textrightarrow Reselect Orientation [O]
        \end{Verbatim}
        \begin{center}
            \includegraphics[scale=0.3]{img/addROI.png}
        \end{center}

        \noindent
        Clicking two points on the Plot Widget defines the orientation axis.
        Note that the angle does not have to be selected perfectly each time. Pretty close will suffice.

    \subsection{Chip Properties Display} 
        Within these chips HotSpotter computes its hotspots --- elliptical
        regions centered on points of interest that HotSpotter automatically
        detects.  Intuitively, the hotspots are loosely analogous to a
        ``fingerprint'' for the chip.  Chips with enough hotspot similarity
        are matched successfully by HotSpotter.

        A chip can be seen by clicking on Chip Table and then selecting a
        chip 

        \begin{center}
          \includegraphics[scale=0.3]{img/viewChip.png}
        \end{center}

        The hotspots' points of interest and elliptical regions can be
        toggled on and off using the menu or hot keys.
        \begin{center}
            \includegraphics[scale=0.2]{img/chipWithEllipse.png}
        \end{center}

    \subsection{Running a Query}
        A Query can be run on the selected chip.

        \begin{Verbatim}[commandchars=\\\{\}]
        Actions \textrightarrow Query [Q]
        \end{Verbatim}
        
\noindent
        This will quickly find similar chips in the database.  The program will
        automatically rank the chips in order of similarity and will highlight
        the portions of the image that it identifies as being most similar.
        Each result will be marked in a different color.

        \begin{center}
            \includegraphics[scale=0.2]{img/query.png}
        \end{center}

\noindent
  Once the user decides based on a query that two or more chips match,
  the actual recording of the match occurs only when the chips that
  match are all given the same name.  This requires an understanding
  of the meaning of the names within HotSpotter, as described next.

   \subsection{IDs, Names, and Recording Matching Results}

   In the Chip Table, users will see a Chip ID, a Chip Name, a Name
   ID, and an Image ID.

\begin{center}
  \includegraphics[width=5.0in]{img/chip-table-with-results.png}
\end{center}

\noindent
  The Image ID is the unique numerical index HotSpotter applies to
  each image.  (Back in the Image Table view, users can see the
  relationship between the Image ID and the image file name.)
  Similarly, the Chip ID is the unique index HotSpotter applies to
  each chip.  (Remember, there can be more than one ROI/chip per image.)
  As HotSpotter does its work and chips are successfully matched,
  users will want to assign names to individual animals and use this
  same name for all chips in which the animal appears.  Initially,
  before an image is recognized, the ``Chip Name'' column value will
  be specified as ``\_\_\_\_'' (four underscores).  This denotes an
  unidentified chip.  A user may double-click on a Chip Name to edit
  it.  Whenever two or more chips are deemed to match, users can edit
  the Chip Name to indicate this.  This is as simple as double
  clicking on the name and assigning it.  Use of copying and pasting
  from one chip name to the other makes this process less prone to
  typing errors.

\section{Additional Tools and Tricks}

Here is a brief discussion of are a few additional tricks and options
for running HotSpotter:
\begin{itemize}
\item \verb+Actions -> Select Next+:
    selects either the next image that does not already have an ROI or the next chip without an
    orientation. 

\item \verb+Options -> Toggle Plot Widget+: 
    shows the results plot in a separate pane.  This is particularly useful for
    resizing the pane when there are many results.

\item \verb+Options -> Edit Preferences+: 
    change the behavior of HotSpotter. For now, these are not very well
    documented and should only be used with extreme care or collaboration with
    the HotSpotter team.

\item \verb+Convenience -> Convert All Images to Chips+: 
    make each image its own ROI and therefore image chip.

\item \verb+Convenience -> Batch Change Name+:  
    change all chip instances of a given name to a new name. This is useful for
    when the same animal is grouped under two different names.

\item \verb+Convenience -> Add Metadata Property+:  
    record metadata as a series of one or more attribute/value pairs for any
    user defined metadata.  HotSpotter will automatically import existing
    metadata from StripeSpotter databases.

\item \verb+Convenience -> Assign Matches Above Threshold+: 
    HotSpotter will automatically run each chip in the database as a query and
    assigns it as a match to any chips whose matching score is above a
    user-define threshold. 

\item \verb+Convenience -> View Data Directory+: 
    Opens the current database directory.

\item \verb+Convenience -> View Source Directory+: (primarily for developer usage)
    Opens the HotSpotter source directory. 

\item \verb+Convenience -> View Internal Directory+: 
    Opens the current database's {\tt .hs\_internals} directory

\item \verb+Matching Experiment+: (primarily for developer usage)
    Runs an experiment to see what matches HotSpotter assigns to each chip.
    Output is written to the database directory. 

\item \verb+Run Name Consistency Experiment+: (primarily for developer usage) 
    Runs an experiment to see if HotSpotter agrees with the current labeling.
    Output is written to the database directory. 

\end{itemize}
  


\section{A Bit of Troubleshooting}

In the event that HotSpotter behaves unexpectedly, the first thing to try is a
restarting the program. If the error persists, the following will fix common
errors: 

\begin{itemize}
    \item \textbf{Delete your preference directory.}\\
        HotSpotter keeps a small set of preference files in the user's  home directory.
        These files remember the last database opened as well as other
        preferences. When updating to new versions these can sometimes cause
        problems. Deleting the {\tt \texttildelow/.hotspotter}\footnote{Note
            that {\tt \texttildelow} denotes the user's home folder} folder may fix some issues.\\

    \item \textbf{Re-Import the Images}\\
        If the images you've imported aren't showing up, you can always re-import
        the images in\\ {\tt user\_database\_dir/images} directory.\\

    \item \textbf{Delete the Computed Directory}\\
        If something looks corrupted or ROIs are being oddly drawn, the user
        should consider deleting the computed directory.  Running  the command 
        {\tt (Convenience \textrightarrow{} View Internal Directory)} will open
        {\tt user\_database\_dir/.hs\_internals} directory. From here the
        {\tt computed} directory may be deleted. This will cause the program
        to recompute all of its data. The user may have to restart HotSpotter.\\


    \item \textbf{Mac OSX 10.8 Gatekeeper}\\ For Mac OSX 10.8 Mountain
      Lion users this app might not run.  This error is due to a
      security feature within Mountain Lion called Gatekeeper.  If the
      app fails to run, please do the following:
        \begin{enumerate}
            \item Go to {\tt System Preferences} --- Click the Apple icon in the menu bar (top-left of the screen) and select {\tt System Preferences} in the drop down menu.
            \item Go to {\tt Security \& Privacy} --- It is located on the top row, entitled {\tt Personal}.
            \item Go to the {\tt General} tab.
            \item Authenticate --- Click on the lock at the bottom-left corner of the screen and subsequently input your computer username and password.
            \item In the bottom half of the {\tt General} tab, there will be the following selection:
                \begin{Verbatim}
                    Allow applications downloaded from:
                    ( ) Mac App Store
                    (X) Mac App Store and identified developers
                    ( ) Anywhere

                \end{Verbatim}

                Select {\tt Anywhere} and subsequently select {\tt Allow From Anywhere} in the drop down warning.

            \item Close the {\tt System Preferences} window.
            \item Install HotSpotter and run it.
            \item To re-enable security after running HotSpotter once,
                repeat the above changes to your preferences, except click on
                \texttt{Mac App Store and identified developers}.
        \end{enumerate}


    \item \textbf{Email the Developer}\\
        If all else fails users should send an email to \developeremail{}.
        Please include a detailed description of the error, the output of the {\tt
            (Convenience \textrightarrow{} Write Logs)}, and what was being done when it
            happened.\\
    \end{itemize}

\section{Source Code Dependencies}

The remainder of this discussion only applies to downloading and
working with the source code instead of the installer packages.

Before executing HotSpotter from the source code users should ensure
that their environment is set up correctly. Primarily, this includes
Python 2.7.3, Qt, and OpenCV, but it also includes several supporting
packages.  Users who want to use HotSpotter without modification
should download the installer package instead.

    \subsection{Windows}
        Install the following dependencies in order.  
        \textbf{The software is untested using 64-bit python. It is preferred to use 32-bit builds of each dependency when specified.}
        \begin{enumerate}
            \item Python 2.7 32-bit
                \begin{enumerate}
                    \item Download:
                        \url{http://www.python.org/download/releases/2.7.5/}
                    \item Install with install packager
                \end{enumerate}

            \item MinGW (C / C++) 
                \begin{enumerate}
                    \item Download: 
                        \url{http://sourceforge.net/projects/mingw/files/latest/download?source=files}
                    \item Install with install packager
                    \item REQUIRED: C and C++ COMPILER
                \end{enumerate}

            \item Qt Library 4.8 32-bit 
                \begin{enumerate}
                    \item Download: \url{http://qt-project.org/downloads}
                    \item Install with install packager
                \end{enumerate}

            \item PyQt4 32-bit 
                \begin{enumerate}
                    \item Download:
                        \url{http://www.riverbankcomputing.com/software/pyqt/download}
                    \item Install with install packager
                \end{enumerate}

            \item NumPy 32-bit 
                \begin{enumerate}
                    \item Download:
                        \url{http://sourceforge.net/projects/numpy/files/NumPy/1.7.1/numpy-1.7.1-win32-superpack-python2.7.exe/download}
                    \item Install with install packager
                \end{enumerate}

            \item matplotlib 32-bit 
                \begin{enumerate}
                    \item Download: \url{http://matplotlib.org/downloads.html}
                    \item Install with install packager
                \end{enumerate}

            \item PIL 32-bit 
                \begin{enumerate}
                    \item Download:
                        \url{http://www.pythonware.com/products/pil/}
                    \item Install with install packager
                \end{enumerate}
        \end{enumerate}
        \;

    \subsection{Mac OSX}

        Install the following dependencies in order.
        \begin{enumerate}

            \item Qt Library 4.8 
                \begin{enumerate}
                    \item Download: \url{http://qt-project.org/downloads}
                    \item Install with install packager
                \end{enumerate}

            \item XQuartz - This dependency is required for FreeType
                (\url{http://www.freetype.org/}) and libpng
                (\url{http://www.libpng.org/pub/png/libpng.html}) 
                \begin{enumerate}
                    \item Download: \url{http://xquartz.macosforge.org}
                    \item Install with install packager
                \end{enumerate}

            \item SIP 
                \begin{enumerate}
                    \item Download:
                        \url{http://riverbankcomputing.co.uk/software/sip/download}
                    \item sudo python configure.py
                    \item sudo make
                    \item sudo make install
                \end{enumerate}

            \item PyQt4 
                \begin{enumerate}
                    \item Download:
                        \url{http://www.riverbankcomputing.com/software/pyqt/download}
                    \item sudo python configure.py
                    \item sudo make
                    \item sudo make install
                \end{enumerate}

            \item NumPy 
                \begin{enumerate}
                    \item Download:
                        \url{http://sourceforge.net/projects/numpy/files/latest/download?source=files}
                    \item sudo python setup.py install
                \end{enumerate}

            \item matplotlib 
                \begin{enumerate}
                    \item Download: \url{http://matplotlib.org/downloads.html}
                    \item sudo python setup.py install
                \end{enumerate}

            \item libjpeg 
                \begin{enumerate}
                    \item Download: \url{http://www.ijg.org/}
                    \item ./configure
                    \item sudo make
                    \item sudo make install
                \end{enumerate}

            \item PIL 
                \begin{enumerate}
                    \item Download:
                        \url{http://www.pythonware.com/products/pil/}
                    \item sudo python setup.py install
                \end{enumerate}
        \end{enumerate}
        \;


\section{Source Code}

    \subsection{License}
HotSpotter is distributed under the GNU General Public License.
\begin{Verbatim} 
    HotSpotter
    Copyright © 2013 Jon Crall, Jason Parham, Chuck Stewart
    Department of Computer Science 
    Rensselaer Polytechnic Institute

    This program is free software: you can redistribute it and/or modify
    it under the terms of the GNU General Public License as published by
    the Free Software Foundation, either version 3 of the License, or
    (at your option) any later version.

    This program is distributed in the hope that it will be useful,
    but WITHOUT ANY WARRANTY; without even the implied warranty of
    MERCHANTABILITY or FITNESS FOR A PARTICULAR PURPOSE.  See the
    GNU General Public License for more details.

    You should have received a copy of the GNU General Public License
    along with this program.  If not, see http://www.gnu.org/licenses/.


\end{Verbatim}

\subsection{Download}
Download the source code here: \url{https://github.com/Erotemic/hotspotter}
\begin{Verbatim}[commandchars=\\\{\}]
git clone git@github.com:Erotemic/hotspotter.git
\end{Verbatim}

Users will also need to check out the tpl submodule.  This can be be
done separately, or by running the command: 
\begin{Verbatim}[commandchars=\\\{\}]
python setup.py configure
\end{Verbatim}
This will also ensure that files have the correct permissions. 

Once the source code has been downloaded the program can be run by using the command:
\begin{Verbatim}[commandchars=\\\{\}]
python main.py
\end{Verbatim}

        %To build the program into a Windows .exe, execute the command:
        %\begin{Verbatim}
        %python setup.py py2exe
        %\end{Verbatim}

        %To build the program into a Mac .app, execute the command:
        %\begin{Verbatim}
        %python setup.py py2app
        %\end{Verbatim}

\subsection{Contribute}
HotSpotter is an open source project. If any tech-savvy users develop a cool
feature or a bug-fix and would like to see it incorporated, send an email with the proposed
patch to \developeremail{} for code review.

\newpage



\end{document}
